\documentclass[UTF8]{ctexbook}
\usepackage{mathtools,amsmath,graphicx,array,booktabs,}
\begin{document}
	
\title{I研堂LaTeX-论文排版讲座大纲}
\author{I研堂}
\date{2019.11.15}
\maketitle
 \tableofcontents 

\chapter{\LaTeX -你的论文排版黑科技}
\subsection{讲座简介}
  \LaTeX,是一个高质量、跨平台、免费的排版系统。通过其排版的文章美观、稳定、通用,因而\LaTeX 在学术界有着广泛的应用。本讲座将主要介绍\LaTeX 的发展历史和基本的使用方法,以及如何使用哈尔滨工业大学毕业论文\LaTeX 模板进行学位论文的写作。
\subsection{讲座对象}全校对于LaTeX感兴趣
的同学
\subsection{讲座大纲}讲座内容分为以下四部分
\begin{enumerate}
	\item \LaTeX 的发展历史及其优势(使同学们初步了解\LaTeX ,及其优势)\\
	\item \LaTeX 排版入门,(介绍\LaTeX 的简单操作,使同学们熟悉运行环境;通过实例讲解\LaTeX 层次、公式、图表的使用,快速掌握使用方法)\\
	\item 学位论文排版(介绍模板的下载和使用、文件模板构成、文献引用)\\
	\item 总结(软件学习方法心得和资料分享推荐)\\
\end{enumerate}
\subsection{讲座要求}本次讲座需要进行实例操作,来听讲座的同学最好自带电脑,并提前依据群内的视频安装\LaTeX 软件。\\

$H(Y | X)=\sum_{x \in \mathcal{X}, y \in \mathcal{Y}} p(x, y) \log \left(\frac{p(x)}{p(x, y)}\right)$
\chapter{ZHCJHVKHJKLN}

\chapter{模式识别简介} 
为啥要学模式识别


呢?
\begin{itemize}
	\item 有用
	\item 好玩 
	
\end{itemize}
为啥要学\LaTeX 呢?
\begin{enumerate}
	\item 确认过眼神
	\item 我是\LaTeX 的人 
\end{enumerate}



\chapter{模式识别的分类}
\begin{enumerate}
	\item 的房产税的
    \item 
\end{enumerate}


\section{监督模式识别}
\subsection{定义}
\subsection{决策树}
决策树中用到的概念之一是熵。一个有$ k $种可能的事件,每种结果的概率为$ P_i,i=1,2,……k, $则信息熵为
\begin{equation}
I=-\sum_{i=1}^{k}P_ilog_2P_i \label{eq:2-1}
\end{equation} 





\section{非监督模式识别}
\subsection{定义}
\subsection{C均值算法}
\chapter{总结}



\section{练习示范区域}

\subsection{测试练习}
Hello,你好呀,\~{}\TeX\~{} !  
Hello, \LaTeX !

\subsection{公式练习区域}


\subsubsection{行间公式}
这是一个测试这是一个测试这是一个测试$ y=x^2+1 0$这是一个测试这是一个测试这是一个测试这是一个测试这是一个测试行间公式$\int^{1}_{10}x\mathrm{dx}=\frac{\int_{1}^{30}y632}{ma^2}$是直接存在于句子当中的,不会单独分行,也不会进行编号,而且高度是与其他文字基本一致的。



\subsubsection{无标号单行公式}
无标号单行公式无标号单行公式无标号单行公式\[ y=\lg2x_i\]无标号单行公式无标号单行公式无标号单行公式无标号单行公式


\subsubsection{是否带编码的练习:}
\begin{equation*}
\label{mass-energy}
E = m{c^2}
\end{equation*}

\begin{equation}
\label{mass-energy2}
F = m{a^2}
\end{equation}



\subsubsection{由$ mythtype $写入的数学公式}
{\textbf{\textit{\underline{}}}}
\[E = {\rm{m}}{{\rm{c}}^2}\sum\nolimits_1^2 {{{\rm{x}}^{{2^{{\rm{f}}{{\rm{g}}_{\rm{c}}}}}}}} \]


\subsubsection{细微区别}
第n个 第$n$个 


\subsubsection{在\LaTeX  中多个行间公式输入}
\begin{align}
&x^{12345}\\
&3x_12345\\
&\frac{x}{2}\\
&\sum_{1}^{2}\frac{x}{2}\\
&\int_{2}^{5}x^2\\
&\mathrm{dx}
\end{align}





\subsection{插入图画练习}
\begin{figure}[htpb] % 图片环境
	\centering
	\includegraphics[width = 0.4\textwidth]{me}
	\caption{图片练习快来练习一下插入属于自己的图片吧}
	\label{fig:me}
\end{figure}
如图\ref{fig:me}所示,这是一张引用的图片,
详情可见第\pageref{fig:me}页。


\begin{figure}[htpb] % 图片环境
	\centering
	\includegraphics[width = 0.3\textwidth]{HIT}
	\caption{我们的学校校徽}
	\label{fig:HIT}
\end{figure}


如图\ref{fig:me}所示,这是一张引用的图片,
详情可见第\pageref{fig:me}页。


% Table generated by Excel2LaTeX from sheet 'Sheet1'
\begin{table}[htbp]
	\centering
	\caption{Add caption}
	\begin{tabular}{r|r}
		\multicolumn{1}{l}{X} & \multicolumn{1}{l}{Y} \\
		1     & 5 \\ \hline
		2     & 7 \\ \hline
		3     & 6 \\
		4     & 8 \\
		&  \\
	\end{tabular}%
	\label{tab:addlabel}%
\end{table}%








\subsubsection{普通表格的制作}


	% Table generated by Excel2LaTeX from sheet 'Sheet1'
	\begin{table}[htbp]
		\centering
		\caption{Add caption}
		\begin{tabular}{|rlr|}
			\toprule
			\multicolumn{1}{|l}{s} & \multicolumn{1}{r}{2} & 7 \\
			\midrule
			\multicolumn{1}{|l}{t} & \multicolumn{1}{r}{4} & 9 \\
			\midrule
			\multicolumn{1}{|l}{d} & \multicolumn{1}{r}{8} &  \\
			\midrule
			6     & \multicolumn{1}{r}{5} & 0 \\
			\midrule
			3     & xz    & 1 \\
			\bottomrule
		\end{tabular}%
		\label{tab:addlabel}%
	\end{table}%
	
	


\begin{tabular}{|r|r|}%代表四列,如果想要添加竖线可以用|
	\hline
	输入&输出\\ 
	\hline
	-2&4\\ \hline
	0&0\\ \hline
	2&4\\\hline
\end{tabular}

\subsection{表格练习示范}
\begin{table}[htbp]
	\caption[table1]{}{符合研究生院绘图规范的表格}{Table$\!$}{Table in agreement of the standard from graduate school}
	\\
	\vspace{0.5em}\centering
	\begin{tabular}{ccccc}
		\toprule[1.5pt]
		$D$(in) & $P_u$(lbs) & $u_u$(in) & $\beta$ & $G_f$(psi.in)\\
		\midrule[1pt]
		5 & 269.8 & 0.000674 & 1.79 & 0.04089\\
		10 & 421.0 & 0.001035 & 3.59 & 0.04089\\
		20 & 640.2 & 0.001565 & 7.18 & 0.04089\\
		\bottomrule[1.5pt]
	\end{tabular}
\end{table}
\begin{table}[]
	\begin{tabular}{|l|r|r|r|l|}
		\hline
		123                     & x            & e            & \multicolumn{1}{l|}{d} & f                      \\ \hline
		\multicolumn{1}{|r|}{8} & \textbf{665} & \textbf{5}   & \textbf{4}             & 3x                     \\ \hline
		2                       & \textbf{4年}  & \textbf{682} & \textbf{7}             & 9                      \\ \hline
		3                       & \textbf{5}   & \textbf{7}   & \textbf{924}           & \multicolumn{1}{c|}{0} \\ \hline
	\end{tabular}
\end{table}

\subsubsection{在excel中导出表格}
% Table generated by Excel2LaTeX from sheet 'Sheet1'
\begin{table}[htbp]
	\centering
	\caption{Add caption}
	\begin{tabular}{rr}
		\multicolumn{1}{l}{x} & \multicolumn{1}{l}{y} \\
		1     & 2 \\
		4     & 3 \\
		5     & 6 \\
	\end{tabular}%
	\label{tab:addlabel}%
\end{table}%\\ 
\newpage





\subsubsection{表格练习题答案}
\begin{table}[htbp]
	\centering
	\caption[table1]{方法总结}
	\begin{tabular}{cc}
		\toprule[1.5pt]
		名称 &分类\\
		\midrule[1pt]
		决策树 &有监督\\
		C均值 &无监督\\
		\bottomrule[1.5pt]
	\end{tabular}
\end{table}


\subsubsection{交叉引用练习}
如果想看这张图片\ref{fig:me}的话,可以去第\pageref{fig:me}页找到		


\end{document}